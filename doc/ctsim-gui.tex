\chapter{The Graphical User Interface}\label{ctsim}\index{ctsim}\index{Graphical interface}%
\setheader{{\it CHAPTER \thechapter}}{}{}{\ctsimheadtitle}{}{{\it CHAPTER \thechapter}}%
\ctsimfooter%


\ctsim\ is the graphical shell for the CTSim project. This shell uses
the \urlref{wxWindows}{http://www.wxwindows.org} library for
cross-platform compatibility. The graphical shell is compatible with
Microsoft Windows, \urlref{GTK}{http://www.gtk.org}, and
\urlref{Motif}{http://www.openmotif.org} graphical environments.

\section{Starting CTSim}
\usage \texttt{ctsim [files to open...]}

You can invoke \ctsim\ by itself on the command line, or include any
number of files that you want \ctsim\ to automatically open. \ctsim\
can open projection files, image files, phantom files, and plot files.

On Microsoft Windows platforms, the simplest way to invoke \ctsim\ is
via the \emph{Start} menu under the \emph{Programs} sub-menu.

\section{Quick Start}\label{IDH_QUICKSTART}\index{Quick Start}
The fastest way to put \ctsim\ through it's basic operation is:
\begin{enumerate}\itemsep=0pt
\item \texttt{File - Create Phantom...} \\
This creates a window with the geometric phantom. Choose the \texttt{Herman} head phantom.
\item \texttt{Process - Rasterize...} \\
This creates an image file of the phantom by converting it from a
geometric definition into a rasterized image. You may use the defaults
shown in the dialog box.
\item \texttt{View - Auto...} \\
Use this command on the new rasterized image window. This will optimize the intensity scale for
viewing the soft-tissue details of the phantom. Select the \texttt{median} center and
a standard deviation factor of \texttt{0.1}.
\item \texttt{Process - Projections...} \\
Use this command on the geometric phantom window. This simulates the collection of x-ray
data. You may use the defaults shown in the dialog box. Additionally, you may wish to turn on
\texttt{Trace - Projections} to watch to x-ray data being simulated.
\item \texttt{Reconstruction - Filtered Backprojection...} \\
Use this command on the projection window. This will reconstruct an image
from the projections. Once again, you may use the defaults shown in the dialog box.
\item \texttt{View - Auto...} \\
Use this command on the new reconstructed image window. This will optimize the intensity scale for
viewing the soft-tissue details of the reconstruction. Select the \texttt{median} center and
a standard deviation factor of \texttt{0.1}.
\item \texttt{Analyze - Compare Images...} \\
Use this command on the rasterized phantom image window. This will bring up a dialog box
asking for the comparison image. Select the reconstruction image that you just made and also select the "Make difference image"
check box. You'll then see the image distance measurements and also a new window with the difference between the rasterized
phantom and the reconstruction.
\item \textbf{That's it!} You have just performed the basic operations with \ctsim. By varying the parameters of the rasterization,
projection, and reconstructions you perform endless computed tomography experiments. \ctsim\ also has many other visualization
and analysis features that you learn more about by reading the manual.
\end{enumerate}

\section{File Types}\index{File types}

\subsection{Phantom}
Besides loading phantom files from the disk, the Herman\cite{HERMAN80} and
Shepp-Logan\cite{SHEPP74} phantoms are built-in to \ctsim. Phantom files can be
read from and written to the disk. Phantom files are stored in a simple
ASCII format. A text editor is required to
create and edit these files.

\subsection{Image}
Image files contain 2-dimensional arrays that store 4-byte floating
point values. Images files can be either real or complex-valued.
Typically, all images are real-valued except for images that have been
processed by Fourier transforms. As you might expect,
complex-valued images are twice the size of real-valued images
since both a real and imaginary component need to be stored. When
complex-valued images are viewed on the screen, only the real
component is displayed.

Images files can    store any number of text labels. \ctsim\ uses
these labels for recording history information regarding
the creation and modifications of images.

\subsection{Projection}
Projection files are created from phantom files via the
projection process. Numerous options are available for the
creation of the these files. The files are stored in a binary
format with cross-platform compatibility on little and big-endian
architectures.

\subsection{Plot}
Plot files are created by \ctsim\ during analysis of image files.
They can be read from and written to the disk. They are stored as ASCII
files for easy cross-platform support and editing.

\section{Global Menu Commands}
These global commands are present on the menus of all windows.

\subsection{File - Create Phantom}\label{IDH_DLG_PHANTOM}\index{Dialog!Create phantom}
This command displays a dialog box showing the phantoms that are pre-programmed
into \ctsim. After selecting one of these phantoms, the new window with that
phantom will be generated. The pre-programmed phantoms are:

\begin{twocollist}
\twocolitem{\textbf{Herman}}{The Herman head phantom\cite{HERMAN80}}
\twocolitem{\textbf{Shepp-Logan}}{The head phantom of Shepp \& Logan\cite{SHEPP74}}
\twocolitem{\textbf{Unit pulse}}{A phantom that has a value of \texttt{1} for the
center of the phantom and \texttt{0} everywhere else.}
\end{twocollist}

\subsection{File - Create Filter}\label{IDH_DLG_FILTER}\index{Dialog!Create filter}
This command displays a dialog box showing the pre-programmed filters
of \ctsim. This command will create a 2-dimensional image of the selected filter.
The center of the filter is at the center of the image.

These filters can be created in their natural frequency domain or in their inverse
spatial domain.

\begin{twocollist}
\twocolitem{\textbf{Filter}}{Selects the filter to generate. The available filters are:

\begin{itemize}\itemsep=0pt
\item $|w|$ Bandlimit
\item $|w|$ Hamming
\item $|w|$ Hanning
\item $|w|$ Cosine
\item $|w|$ Sinc
\item Shepp
\item Bandlimit
\item Sinc
\item Hamming
\item Hanning
\item Cosine
\item Triangle
\end{itemize}
}
\twocolitem{\textbf{Domain}}{Selects either the \texttt{Frequency} or \texttt{Spatial} domain. The filters have the
frequency domain as their natural domain. The spatial domain is obtained either analytically or performing
an inverse Fourier transformation.}
\twocolitem{\textbf{X Size}}{Number of columns in the output image.}
\twocolitem{\textbf{Y Size}}{Number of rows in the output image.}
\twocolitem{\textbf{Hamming Parameter}}{ This parameter adjusts the smoothing of the Hamming
filter and can range from \texttt{0} to \texttt{1}.
At a setting of \texttt{1}, the Hamming filter is the same as the bandlimit filter.
At a setting of \texttt{0.54}, the Hamming filter is the same as the Hanning
window.}
\twocolitem{\textbf{Bandwidth}}{Sets the bandwidth of the filter.}
\end{twocollist}

\begin{twocollist}
\twocolitem{\textbf{Axis (input) Scale}}{Sets the scale for the filter input. By default, the input to the filter is
the distance in pixels from the center of the image. By changing this value, one can set a scale the input to the filter.
For example, if the output image is \texttt{101} by \texttt{101} pixels and thus the center of the image is at \texttt{(50,50)}, then a pixel
lying at point \texttt{100,50} would be 50 units from the center of the filter. By applying an \texttt{Axis scale} of
\texttt{0.1}, then that point would be scaled to 5 units from the center of the filter.}
\twocolitem{\textbf{Filter (output) Scale}}{Multiplies the output of the filter by this amount. By default, the filter has a maximum
value of \texttt{1}.}
\end{twocollist}

\subsection{File - Import}\label{IDH_DLG_IMPORT}
This command allows the importing of non-\ctsim\ file formats into
\ctsim.  \texttt{PPM} and \texttt{PNG} formats will be read into an
imagefile window. Color images will be converted to grayscale. If a
\texttt{DICOM} library was linked in with your version of \ctsim,
then you can also import \texttt{DICOM} projection files and image
files.

\subsection{File - Preferences}\label{IDH_DLG_PREFERENCES}\index{Dialog!Preferences}
This command displays a dialog box that allows users to control
the behavior of \ctsim. These options are saved across \ctsim\ sessions.
Under Microsoft Windows environments, they are stored in the registry.
On UNIX and Linux environments, they are stored in the user's home
directory with the filename of \texttt{.ctsim}.

\begin{twocollist}
\twocolitem{\textbf{Advanced options}}{This option is initially turned off in new installations.
These advanced options are not required for normal simulations. When \texttt{Advanced
Options} is set, \ctsim\ will display more options during scanning of phantoms and
the reconstruction of projections.}

\twocolitem{\textbf{Ask before closing new documents}}{This option is initially turned on in
new installations. With this option set, \ctsim\ will ask before closing
documents that have been modified or never saved on disk. By turning off
this option, \ctsim\ will never ask if you want to save a file -- you'll
be responsible for saving any files that you create.}

\twocolitem{\textbf{Verbose logging}}{This option is initially turned off in
new installations. With this option set, \ctsim\ will log more events than
usual. There extra events are not important for viewing with typical operation of
\ctsim.}

\twocolitem{\textbf{Show startup tips}}{This option is initially turned on in
new installations. With this option set, \ctsim\ will display
helpful tips when \ctsim\ is started.}

\twocolitem{\textbf{Run background tasks}}{This option is initially turned off in
new installations. With this option set, \ctsim\ execute lengthy calculations in the
background. A background window will appear when processes are running in the background
and will disappear when no background processes are executing. This background window shows
the status and progress of all background processes. 
\textbf{NOTE:} Due to limitations of \texttt{wxWindows}, this function is
only supported on Microsoft Windows.
}

\end{twocollist}

\subsection{File - Open}
This command opens a file section dialog box. Of special consideration
is the \texttt{File Type} combo box on the bottom of the dialog. You need
to set the this combo box to the type of file that you wish to open.

\subsection{File - Save}
This command saves the contents of the active window. If the window hasn't
been named, a dialog box will open asking for the file name to use.

\subsection{File - Close}
As one would expect, this closes the active window. If the contents of the window have not been
saved and the \emph{Advanced Preferences} option \texttt{Ask before closing new documents}
is turned on, then you will be prompted if decide if you want to save the contents of the window
prior to closing.

\subsection{File - Save As}
Allows the saving of the contents of the active window to any file name.

\subsection{Help - Contents}
This command displays the online help.

\subsection{Help - Tips}
This command displays a dialog with tips on using \ctsim.

\subsection{Help - Quick Start}
This command displays a recommend approach to helping new users learn to use \ctsim.

\subsection{Help - About}
This command shows the version number and operating environment of \ctsim.


\section{Phantom Menus}

\subsection{Properties}
Displays the properties of a phantom which includes:

\begin{itemize}\itemsep=0pt
\item Overall dimensions of a phantom
\item A list of all component phantom elements
\end{itemize}

\subsection{Process - Rasterize}\label{IDH_DLG_RASTERIZE}\index{Dialog!Rasterize}
This creates an image file from a phantom. Technically, it
converts the phantom from a vector (infinite resolution) object
into a 2-dimension array of floating-point pixels. The parameters
to set are:

\begin{twocollist}
\twocolitem{\textbf{X size}}{Number of columns in image file}
\twocolitem{\textbf{Y size}}{Number of rows in image file}
\twocolitem{\textbf{Samples per pixel}}{Numbers of samples taken
per pixel in both the x and y directions. For example, if the
\texttt{Samples per pixel} is set to \texttt{3}, then for every
pixel in the image file 9 samples \latexonly{($3\times3$)}\latexignore{(3 x 3)}
are averaged.}
\end{twocollist}

\subsection{Process - Projections}\label{IDH_DLG_PROJECTIONS}\index{Dialog!Projections}
This command creates a projection file from a phantom. The options
available when collecting projections are:

\begin{twocollist}
\twocolitem{\textbf{Geometry}}{Sets the scanner geometry. The available geometries are:
  \begin{itemize}\itemsep=0pt
    \item \texttt{Parallel}
    \item \texttt{Equiangular}
    \item \texttt{Equilinear}
  \end{itemize}}
\twocolitem{\textbf{Number of detectors}}{Sets the number of
detectors in the detector array.}

\twocolitem{\textbf{Number of views}}{Sets the number of views
to collect.}

\twocolitem{\textbf{Samples per detector}}{Sets the number of
samples collected for each detector.}

\twocolitem{\textbf{View Ratio}}{Sets the field of view as a ratio
of the diameter of the phantom.  For normal scanning, use a value of
\texttt{1.0}.}

\twocolitem{\textbf{Scan Ratio}}{Sets the length of scanning as a
ratio of the view diameter. For normal scanning, use a value of \texttt{1.0}.}

\twocolitem{\textbf{Focal length ratio}}{Sets the distance between the
radiation source and the center of the phantom as a
ratio of the radius of the phantom. For parallel geometries, a value
of \texttt{1.0} is optimal. For other
geometries, this should be at least \texttt{2.0} to avoid artifacts.}
\end{twocollist}

\textbf{Advanced Options}

These options are visible only if \emph{Advanced Options} has been
selected in the \texttt{File - Preferences} dialog. These parameters
default to optimal settings and don't need to be adjusted except
by expert users.

\begin{twocollist}
\twocolitem{\textbf{Rotation Angle}}{Sets the rotation amount as a
fraction of a circle. For parallel geometries use a rotation angle of \texttt{0.5}
and for equilinear and equiangular geometries use a rotation angle
of \texttt{1}. Using any other rotation angle will lead to artifacts.}
\end{twocollist}



\section{Image Menus}
\subsection{File - Properties}
Properties of image files include:
\begin{itemize}\itemsep=0pt
  \item Whether the image is real or complex-valued.
  \item Numeric statistics (minimum, maximum, mean, median, mode, and standard deviation).
  \item History labels (text descriptions of the processing for this image).
\end{itemize}

\subsection{File - Export}\label{IDH_DLG_EXPORT}\index{Image!Export}
This command allows for exporting image files to a standard
graphics file format. This is helpful when you want to take an
image and import it into another application. The current
\helprefn{intensity scale}{intensityscale} is used when exporting
the file. The supported graphic formats are:

\begin{twocollist}
\twocolitem{\textbf{PNG}}{Portable Network Graphics format. This uses 8-bits or
256 shades of gray.}
\twocolitem{\textbf{PNG-16}}{This is a 16-bit version of PNG which allows for
65536 shades of gray.}
\twocolitem{\textbf{PGM}}{Portable Graymap format. This is a common format used on
UNIX systems.}
\twocolitem{\textbf{PGM ASCII}}{ASCII version of PGM.}
\end{twocollist}


\subsection{View}
\subsubsection{Intensity Scale}\label{intensityscale}\index{Intensity scale}
These commands are used change the intensity scale for viewing the image.
These commands do not change the image data. When the minimum value is
set, then the color pure black is assigned to that image value. Similarly,
when the maximum value is set, the the color pure white is assigned to that
image value.

Changing the intensity scale is useful when examining different image features.
In clinical medicine, the intensity scale is often changed to examine bone
(high value) versus soft-tissue (medium value) features.

\subsubsection{Set}\label{IDH_DLG_MINMAX}
This command displays a dialog box that sets the lower
and upper values to display.

\subsubsection{Auto}\label{IDH_DLG_AUTOSCALE}\index{Auto scale}
This command displays a dialog box that allows \ctsim\ to automatically
make an intensity scale. The parameters that \ctsim\ needs to make this
automatic scale are:

\begin{twocollist}
\twocolitem{\textbf{Center}}{This sets the center of the intensity scale. Currently,
\ctsim\ allows you to use either the mean, mode, or median of the image
as the center of the intensity scale.}

\twocolitem{\textbf{Width}}{This sets the half-width of the intensity scale. The half-width
is specified as a multiple of the standard deviation.}
\end{twocollist}

As an example, if \texttt{median} is selected as the center and
\texttt{0.5} is selected as the width, the the minimum value will
be \latexonly{$median - 0.5 \times standardDeviation$}\latexignore{\emph{median - 0.5 x standardDeviation}}
and the maximum value will be \latexonly{$median + 0.5 \times standardDeviation$.}\latexignore{\emph{
median + 0.5 x standardDeviation}.}

\subsubsection{Full}
This command resets the intensity scale to the full scale of the image.

\subsection{Image}
These commands create a new image based upon the current image,
and for some commands, also upon a comparison image.

\subsubsection{Add, Subtract, Multiply, Divide}
These are simple arithmetic operations. \ctsim\ will display a dialog
box showing all of the currently opened image files that are the
same size as the active image. After the selection of a compatible image,
\ctsim\ will perform the arithmetic operation on the two images and
make a new result image.

\subsubsection{Image Size}
This command will generate a new image based on the current image. The new
image can be scaled to any size. A dialog
appears asking for the size of the new image. Bilinear interpolation
is used when calculating the new image.

\subsubsection{3-D Conversion}
This command generates a 3-dimensional view of the current phantom. This view can be
rotated in three dimensions. The left and right arrow control the z-axis
rotation and the up and down arrows control the x-axis rotation. The y-axis
rotation is controlled by the \texttt{T} and \texttt{Y} keys. Other options
are presented on the \texttt{View} menu and include:

\begin{itemize}\itemsep=0pt
\item Surface plot versus wireframe plot.
\item Smooth shading versus flat shading.
\item Lighting on or off.
\item Color scale on or off.
\end{itemize}

\subsection{Filter}\index{Image!Filter}
These commands filter and modify the image

\subsubsection{Arithmetic}
These commands operate on the image on a pixel-by-pixel basis. The commands
support both real and complex-valued images. The available arithmetic commards are:

\begin{twocollist}
  \twocolitem{\textbf{Invert}}{Negate pixel values.}
  \twocolitem{\textbf{Log}}{Take natural logrithm of pixel values.}
  \twocolitem{\textbf{Exp}}{Take natural exponent of pixel values.}
  \twocolitem{\textbf{Square}}{Take square of pixel values.}
  \twocolitem{\textbf{Square root}}{Take square root of pixel values.}
\end{twocollist}


\subsubsection{Frequency Based}
These commands perform Fourier and inverse Fourier transformations of
images. By default, the transformations will automatically convert
images between Fourier to natural orders as expected. For example, \texttt{2-D FFT}
will transform the points into natural order after the Fourier transform.
Similarly the inverse, \texttt{2-D IFFT}, will reorder the points from
natural order to Fourier order before applying the inverse Fourier transformation.

As you would expect, images that undergo frequency filtering will be complex-valued
after than filtering. Only the real component is shown by \ctsim. However, \ctsim\ does
have options for converting a complex-valued image into a real-valued image via
the \texttt{Magnitude} and \texttt{Phase} filtering commands.

The available frequency-based filtering commards are:

\begin{itemize}\itemsep=0pt
\item 2-D FFT
\item 2-D IFFT
\item FFT Rows
\item IFFT Rows
\item FFT Columns
\item IFFT Columns
\item 2-D Fourier
\item 2-D Inverse Fourier
\item Shuffle Fourier to Natural Order
\item Shuffle Natural to Fourier Order
\item Magnitude
\item Phase
\end{itemize}

\subsection{Analyze - Plot}
The commands plot rows and columns of images. There are commands
that perform FFT transformations prior to plotting. To select
the row or column to plot, click the left mouse button over the
desired cursor point.

The available plot commands are:
\begin{itemize}\itemsep=0pt
\item Plot Row
\item Plot Column
\item Plot Histogram
\item Plot FFT Row
\item Plot FFT Col
\end{itemize}

\subsection{Analyze - Compare}\label{IDH_DLG_COMPARISON}\index{Image!Comparison}
This command performs statistical comparisons between two images. An option
also exists for generating a difference image from the two input images.

The three distance measures reported are:
\begin{itemize}\itemsep=0pt
\item[] \textbf{$d$}\quad The normalized root mean squared distance measure.
\item[] \textbf{$r$}\quad The normalized mean absolute distance measure.
\item[] \textbf{$e$}\quad The worst case distance measure over a \latexonly{$2\times2$}\latexignore{\emph{2 x 2}} pixel area.
\end{itemize}

There are also commands for comparison plotting of rows and columns from two images.
This is quite helpful when comparing a phantom to a reconstruction. As with plotting
of rows and columns, click the left mouse button over the desired cursor point to
choose which row and column to plot.


\section{Projection Menus}

\subsection{File - Properties}
The displayed properties include:

\begin{itemize}\itemsep=0pt
\item Number of detectors in the projections.
\item Number of views.
\item The parameters used when generating the projections from the phantom.
\end{itemize}

\subsection{Process - Convert Rectangular}
The commands takes the projection data and creates an image file using
the projection data.

\subsection{Process - Convert Polar}\label{IDH_DLG_POLAR}\index{Polar conversion}
This command creates an image file with the polar conversion of the projection data.
The parameters to set are:

\begin{twocollist}
\twocolitem{\textbf{X Size}}{Number of columns in output image.}
\twocolitem{\textbf{Y Ssize}}{Number of rows in output image.}
\twocolitem{\textbf{Interpolation}}{Selects the interpolation method.
Currently, the \texttt{bilinear} option provides the highest
quality interpolation.}
\end{twocollist}

\subsection{Convert - Convert FFT Polar}
The parameters for this option are the same as the \helprefn{Convert
Polar Dialog}{convertpolardialog}. For this command, though, the
projections are Fourier transformed prior to conversion to polar
image.

\subsection{Convert - Interpolate to Parallel}
This command filters divergent projection data (equiangular or
equilinear) and interpolates (or rebins) to estimate the projection
data if the projections had been collected with parallel geometry.

\subsection{Analyze - Plot Histogram}
Plots a histogram of projection data attenuations.

\subsection{Analyze - Plot T-Theta Sampling}
Plots a 2-dimensional scattergram showing the \texttt{T} and
\texttt{Theta} values for each data point in the projection data.
This is especially instructive when scanning with divergent
geometries and the scan ratio is close to \texttt{1}.

\subsubsection{Theta Range}\label{IDH_DLG_THETA_RANGE}
This dialog box allows the constraint of Theta values for the
T-Theta Sampling scattergram. 

\subsection{Reconstruct - Filtered Backprojection}\label{IDH_DLG_RECONSTRUCTION}\index{Dialog!Reconstruction}
This command displays a dialog to set the parameters for reconstructing an image from projections
using the filtered backprojection technique. The parameters available are:

\begin{twocollist}
\twocolitem{\textbf{Filter}}{Selects the filter to apply to each
projection. To properly reconstruct an image, this filter should
consist of the the absolute value of distance from zero
frequency optionally multiplied by a smoothing filter. The optimal
filters to use are:
\begin{itemize}\itemsep=0pt
\item \texttt{abs\_bandlimit}
\item \texttt{abs\_hamming}
\item \texttt{abs\_hanning}
\item \texttt{abs\_cosine}
\end{itemize}
} \twocolitem{\textbf{Hamming parameter}}{Sets the alpha level for
Hamming window. This parameter adjusts the smoothing of the Hamming
filter and can range from \texttt{0} to \texttt{1}.
At a setting of \texttt{1}, the Hamming filter is the same as the bandlimit filter.
At a setting of \texttt{0.54}, the Hamming filter is the same as the Hanning
window.}
\twocolitem{\textbf{Filter Method}}{Selects the filtering method.
For large numbers of detectors, the FFT-based filters are preferred whereas for
smaller numbers of detectors \texttt{convolution} can be
faster. When \emph{Advanced Options} have been turned off, this menu only shows
the two basic choices: \texttt{convolution} and \texttt{FFT}. However, when
\emph{Advanced Options} have been turned on, additional selections are available as
discussed in the next section.
}
\twocolitem{\textbf{Interpolation}}{Interpolation technique during backprojection.
\texttt{cubic} has optimal quality when the
data is smooth. Smooth data is obtained by taking many projections and/or
using a smoothing filter. In the absence of smooth data, \texttt{linear} gives better results and
is also many times faster than cubic interpolation.

\begin{itemize}\itemsep=0pt
\item \texttt{nearest} - No interpolation, selects nearest point.
\item \texttt{linear} - Uses fast straight line interpolation.
\item \texttt{cubic} - Uses cubic interpolating polynomial.
\end{itemize}
}
\end{twocollist}

\textbf{Advanced Options}

These options are visible only if \emph{Advanced Options} has been
selected in the \texttt{File - Preferences} dialog. These parameters
default to optimal settings and don't need to be adjusted except
by expert users.

\begin{twocollist}
\twocolitem{\textbf{Filter Method}}{Selects the filtering method.
The general comments about this parameter given the previous section still apply.
With \emph{Advanced Options} on, the full set of filter methods are available:
\begin{itemize}\itemsep=0pt
\item \texttt{convolution}
\item \texttt{fourier} - Uses simple Fourier transform.
\item \texttt{fourier-table} - Optimizes Fourier transform by precalculating trigometric functions.
\item \texttt{fftw} - Uses complex-valued Fourier transform with the \emph{fftw} library.
\item \texttt{rfftw} - Uses optimized real/half-complex Fourier transform.
\end{itemize}
}
\twocolitem{\textbf{Backprojection}}{Selects the backprojection
technique. A setting of \texttt{idiff} is optimal.
\begin{itemize}\itemsep=0pt
\item \texttt{trig} - Use trigometric functions at each image point.
\item \texttt{table} - Use precalculated trigometric tables.
\item \texttt{diff} - Use difference method to iterate within image.
\item \texttt{idiff} - Use integer iteration techique.
\end{itemize}
}

\twocolitem{\textbf{Filter Generation}}{Selects the filter
generation. With convolution, \texttt{direct} is the proper method
to select. With any of the frequency methods,
\texttt{inverse-fourier} is the best method.
\begin{itemize}\itemsep=0pt
\item \texttt{direct}
\item \texttt{inverse-fourier}
\end{itemize}
}

\twocolitem{\textbf{Zeropad}}{Zeropad factor when using
frequency-based filtering. A setting of \texttt{1} is optimal whereas
a setting of \texttt{0} disables zero padding. Settings greater than \texttt{1}
perform larger amounts of zero padding but without any significant benefit.}

\twocolitem{\textbf{ROI}}{These four settings control the
\textit{region of interest} for the reconstruction. The default values
match the dimensions of the entire phantom. By constraining the ROI to
be a smaller square, the reconstruction will be magnified.}

\end{twocollist}

\subsection{Reconstruct - Filtered Backprojection (Rebin to Parallel)}\label{IDH_DLG_RECONSTRUCTION_REBIN}\index{Dialog!ReconstructionRebin}
The command reconstructs the projection data via filtered backprojection
as described above. As opposed to the above command, this command also
rebins divergent projection data to parallel prior to reconstruction.
This greatly speeds reconstruction of divergent geometry projections.

\section{Plot Menus}
\subsection{File - Properties}
The displayed properties include:

\begin{itemize}\itemsep=0pt
\item the number of curves in the plot and the number of points per curve.
\item the EZPlot commands used to format the plot are displayed.
\item history labels from the originating image(s) and the plot function.
\end{itemize}

\subsection{View Menu}
These commands set the scaling for the y-axis. They are analogous
to the options used for setting the intensity scale for images.

\subsubsection{Set}
This command sets the upper and lower limits for the y-axis.

\subsubsection{Auto}
This command automatically sets the upper and lower limits for the
y-axis. Please refer to the image file \helpref{\texttt{View - Auto}}{IDH_DLG_AUTOSCALE}
documentation for the details.

\subsubsection{Full}
The command resets the upper and lower limits of the y-axis to the
minimum and maximum values of the curves.
