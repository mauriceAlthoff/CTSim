\chapter{The Command Line Interface}\label{ctsimtext}\index{ctsimtext}\index{Command line interface}
\setheader{{\it CHAPTER \thechapter}}{}{}{\ctsimheadtitle}{}{{\it CHAPTER \thechapter}}%
\ctsimfooter%

\ctsimtext\ is the master shell for all of the command-line tools. The
command-line tools can perform most of the functions of the graphical
shell. These command-line tools are especially appropriate for use on
systems without graphical capability or for batch processing, shell scripting,
and parallel processing.

\section{Starting ctsimtext}
\ctsimtext\ can be invoked via three different
methods.
\begin{enumerate}\itemsep=3pt
\item \ctsimtext\ can executed without any parameters. In that case,
\ctsimtext\ offers a command-line to enter the function-names and
their parameters. The output of the command is displayed. Further
commands may be given to \ctsimtext. The shell is exited by the
\texttt{quit} command. \ctsimtext\ uses the
\urlref{readline}{http://www.gnu.org} library on UNIX and Linux platforms
to provide command history processing.

\item \ctsimtext\ can also be called to
execute a single command. This is especially useful for batch
files containing multiple \ctsimtext\ commands. This is invoked by
calling\\ \texttt{ctsimtext function-name parameters...}.

\item Using operating systems that support soft or hard linking of
files (such as UNIX and Linux), the executable file \ctsimtext\ can
be linked to the function names. This is automatically done by
the installation program and the \texttt{rpm} manager. Thus, to use \ctsimtext\ with the
function name \texttt{pjrec}, the below command can be executed:\\
\hspace*{1.5cm}\texttt{pjrec parameters...} \\
as a shortcut to the equivalent command \\
\hspace*{1.5cm}\texttt{ctsimtext pjrec parameters...}

\end{enumerate}

\section{Parallel Processing}\label{ctsimtextlam}\index{Parallel processing}\index{MPI}\index{LAM}
\ctsimtext\ can distribute it's processing over a cluster. Specifically,
\ctsimtext\ supports the \urlref{LAM}{http://www.mpi.nd.edu/lam} version of
the MPI environment. On platforms with LAM installed, a parallel version of
\ctsimtext\ is created. The name of this program is \texttt{ctsimtext-lam}.
The functions that take advantage of the parallel processing are:

\begin{itemize}\itemsep=0pt
\item \texttt{phm2if}
\item \texttt{phm2pj}
\item \texttt{pjrec}
\end{itemize}

This parallel processing version has been tested with excellent results on
a 16-CPU \urlref{Beowulf}{http://www.beowulf.org} cluster.


\section{if1}\label{if1}\index{if1}%
Performs math functions on a single image. The commands works with
both real and complex-valued images.

\usage
\texttt{if1 input-filename output-filename [options...]}

\textbf{Options}

\begin{twocollist}
  \twocolitem{\doublehyphen{invert}}{Negate pixel values.}
  \twocolitem{\doublehyphen{log}}{Take natural logrithm of pixel values.}
  \twocolitem{\doublehyphen{exp}}{Take natural exponent of pixel values.}
  \twocolitem{\doublehyphen{sqr}}{Take square of pixel values.}
  \twocolitem{\doublehyphen{sqrt}}{Take square root of pixel values.}
\end{twocollist}

\section{if2}\label{if2}\index{if2}%
Performs math functions on a two images. The command works with both
real and complex-valued images.

\usage
\texttt{if2 input-filename1 input-filename2 output-filename [options...]}

\textbf{Options}

\begin{twocollist}
  \twocolitem{\doublehyphen{add}}{Add the two images.}
  \twocolitem{\doublehyphen{sub}}{Subtract the two images.}
  \twocolitem{\doublehyphen{multiply}}{Multiply the two images.}
  \twocolitem{\doublehyphen{divide}}{Divide the two images.}
  \twocolitem{\doublehyphen{comp}}{Statistically compare the two images. The standard
  \helpref{three distance measurements}{conceptimagecompare} are reported.}
  \twocolitem{\doublehyphen{column-plot n}}{Plot the values of a particular column. The plot file is saved to disk.}
  \twocolitem{\doublehyphen{row-plot n}}{Plot the values of a particular row. The plot file is saved to disk.}
\end{twocollist}

\section{ifexport}\label{ifexport}\index{ifexport}%
Export an image file to a standard graphics file.

\usage
\texttt{ifexport input-filename output-filename [options...]}

\textbf{Options}

\begin{twocollist}
  \twocolitem{\doublehyphen{format}}{
  \begin{itemize}\itemsep=0pt
    \item \texttt{png} - Portable network graphics format. This is the default output format.
    \item \texttt{png16} - 16-bit PNG format.
    \item \texttt{pgm} - Portable graymap format, binary format.
    \item \texttt{pgmasc} - ASCII PGM format.
  \end{itemize}}
  \twocolitem{\doublehyphen{center}}{Set center of intensity window.
  \begin{itemize}\itemsep=0pt
    \item \texttt{median}
    \item \texttt{mode}
    \item \texttt{mean}
  \end{itemize}}
  \twocolitem{\doublehyphen{auto}}{Set half-width of intensity window as a multiple of the standard deviation.
  \begin{itemize}\itemsep=0pt
     \item \texttt{full}
     \item \texttt{std0.1}
     \item \texttt{std0.5}
     \item \texttt{std1}
     \item \texttt{std2}
     \item \texttt{std3}
  \end{itemize}}
  \twocolitem{\doublehyphen{scale}}{Set size of output image. A value of
  \texttt{1} is default and creates an output image the same size as the input image. A value of \texttt{2} will double the
  size of the output image.}
  \twocolitem{\doublehyphen{min}}{Set the minimum intensity value.}
  \twocolitem{\doublehyphen{max}}{Set the maximum intensity value.}
\end{twocollist}

\section{ifinfo}\label{ifinfo}\index{ifinfo}%

Displays information about an image file. By default, history labels and image statistics are displayed.

\usage
\texttt{ifinfo input-filename [options...]}

\textbf{Options}

\begin{twocollist}
  \twocolitem{\doublehyphen{no-labels}}{Suppress history labels.}
  \twocolitem{\doublehyphen{no-stats}}{Suppress image statistics.}
\end{twocollist}

\section{phm2pj}\label{phm2pj}\index{phm2pj}%
Simulates collection of X-rays data (projections) around a phantom object.

\usage
\texttt{phm2pj projection-filename number-detectors number-views [options...]}

\textbf{Options}

\begin{twocollist}
\twocolitem{\doublehyphen{phantom}}{Select a standard phantom.
\begin{itemize}\itemsep=0pt
\item \texttt{herman}
\item \texttt{shepp-logan}
\item \texttt{unit-pulse}
\end{itemize}
}
\twocolitem{\doublehyphen{phmfile}}{Reads a user-created phantom file.}

\twocolitem{\doublehyphen{geometry}}{Sets the scanner geometry. Valid values are:
  \begin{itemize}\itemsep=0pt
    \item \texttt{parallel}
    \item \texttt{equiangular}
    \item \texttt{equilinear}
  \end{itemize}
}

\twocolitem{\doublehyphen{nray}}{ Number of samples per each detector}

\twocolitem{\doublehyphen{rotangle}}{The rotation angle as a fraction of a circle.
For parallel geometries use a rotation angle of \texttt{0.5} and for equilinear and equiangular
geometries use a rotation angle of \texttt{1}. The default is to use to
appropriate rotation angle based on the geometry.}

\twocolitem{\doublehyphen{view-ratio}}{Sets the field of view as a ratio of the diameter of the phantom.
    For normal scanning, the default value of \texttt{1.0} is optimal.}

\twocolitem{\doublehyphen{scan-ratio}}{Sets the length of scanning as a ratio of the view diameter.
    For normal scanning, the default value of \texttt{1.0} is optimal.}

\twocolitem{\doublehyphen{focal-length}}{Sets the distance between the radiation source
 and the center of the object as a ratio of the radius of the object.
   For parallel geometries, a value of \texttt{1.0} is optimal. For other
    geometries, this should be at least \texttt{2.0} to avoid artifacts. The default value is \texttt{2}.}
\end{twocollist}


\section{phm2if}\label{phm2if}\index{phm2if}%
Generates a raster image file based on a phantom.

\usage
\texttt{phm2if phantom-filename image-filename x-image-size y-image-size [options...]}

\textbf{Options}

\begin{twocollist}
  \twocolitem{\doublehyphen{nsamples}}{Number of samples in x and y directions per pixel}
  \twocolitem{\doublehyphen{view-ratio}}{Sets the view ratio. For normal scanning,
  the default value of \texttt{1.0} is optimal.}
\end{twocollist}

\section{pj2if}\label{pj2if}\index{pj2if}%
Convert a projection file into an image file where each row of the
image file contains the projection data from a single view.

\usage
\texttt{pj2if projection-filename image-filename [options...]}

\textbf{Options}

\begin{twocollist}
\twocolitem{\doublehyphen{dump}}{Print all projection data to the console.}
\end{twocollist}

\section{pjinfo}\label{pjinfo}\index{pjinfo}%
Displays information about a projection file.

\usage
\texttt{pjinfo projection-filename [options...]}

\textbf{Options}

\begin{twocollist}
  \twocolitem{\doublehyphen{binaryheader}}{Dump the binary header to the standard output.
  This option is only used when manually creating a composite projection file from
  several different projection files.}
  \twocolitem{\doublehyphen{binaryview}}{Dump binary view data to the standard output.
  This option is only used when manually creating a composite projection file from
  several different projection files.}
  \twocolitem{\doublehyphen{startview}}{Sets starting view to display. Default is \texttt{0}.}
  \twocolitem{\doublehyphen{endview}}{Sets ending view to display. Default is the last view.}
  \twocolitem{\doublehyphen{dump}}{Print all projection data to the console.}
\end{twocollist}

\section{pjrec}\label{pjrec}\index{pjrec}%
Reconstructs the interior of an object from a projection file.

\usage
\texttt{pjrec projection-filename image-filename image-cols image-rows [options...]}

\textbf{Options}

\begin{twocollist}
\twocolitemruled{\textbf{Parameter}}{\textbf{Options}}
\twocolitem{\doublehyphen{filter}}{Selects which filter to apply to
each projection. To properly reconstruct an image, this filter should
consist of the the absolute value of distance from zero
frequency optionally multiplied by a smoothing filter. The optimal
filters to use are:
\begin{itemize}\itemsep=0pt
\item \texttt{abs\_bandlimit}
\item \texttt{abs\_cosine}
\item \texttt{abs\_hamming}
\item \texttt{abs\_hanning}
\end{itemize}
} \twocolitem{\doublehyphen{filter-parameter}}{Sets the alpha level
for Hamming window.  This parameter adjusts the smoothing of the Hamming
filter and can range from \texttt{0} to \texttt{1}.
At a setting of \texttt{1}, the Hamming filter is the same as the bandlimit filter.
At a setting of \texttt{0.54}, the Hamming filter is the same as the Hanning
window.}

\twocolitem{\doublehyphen{filter-method}}{Selects the filtering
method. For large numbers of detectors, \texttt{rfftw} is optimal.
For smaller numbers of detectors, \texttt{convolution} might be a
bit faster.
\begin{itemize}\itemsep=0pt
\item \texttt{convolution}
\item \texttt{fourier} - Uses simple Fourier transform.
\item \texttt{fourier-table} - Optimizes Fourier transform by precalculating trigometric functions.
\item \texttt{fftw} - Uses complex-valued Fourier transform with the \emph{fftw} library.
\item \texttt{rfftw} - Uses optimized real/half-complex Fourier transform.
\end{itemize}
}
\end{twocollist}

\begin{twocollist}
\twocolitem{\doublehyphen{filter-generation}}{Selects the filter
generation. With convolution, \texttt{direct} is the proper method
to select. With any of the frequency methods,
\texttt{inverse-fourier} is the best method.
\begin{itemize}\itemsep=0pt
\item \texttt{direct}
\item \texttt{inverse-fourier}
\end{itemize}
}

\twocolitem{\doublehyphen{interpolation}}{Interpolation technique during backprojection.
\texttt{cubic} has optimal quality when the
data is smooth. Smooth data is obtained by taking many projections and/or
using a smoothing filter. In the absence of smooth data, \texttt{linear} gives better results and
is many times faster than cubic interpolation.

\begin{itemize}\itemsep=0pt
\item \texttt{nearest} - No interpolation, selects nearest point.
\item \texttt{linear} - Uses fast straight line interpolation.
\item \texttt{cubic} - Uses cubic interpolating polynomial.
\end{itemize}
}

\twocolitem{\doublehyphen{backprojection}}{Selects the
backprojection technique. A setting of \texttt{idiff} is optimal.
\begin{itemize}\itemsep=0pt
\item \texttt{trig} - Use trigometric functions at each image point.
\item \texttt{table} - Use precalculated trigometric tables.
\item \texttt{diff} - Use difference method to iterate within image.
\item \texttt{idiff} - Use integer iteration technique.
\end{itemize}
}

\twocolitem{\doublehyphen{zeropad}}{Zeropad factor. A setting of
\texttt{1} is optimal whereas a zeropad of \texttt{0} performs no zero padding.
Settings greater than \texttt{1} perform additional zero padding, but without
any significant output difference.}

\end{twocollist}
